\documentclass{article}

\title{Ćwiczenia 2: sumy / zadanie 1.a)}
\author{Paweł Balawender}
\date{Marzec 2021}

\usepackage{geometry}
\geometry{
    a4paper,
    total={170mm,257mm},
    left=10mm,
    right=10mm,
    top=10mm
}

\usepackage{commath}
\usepackage[utf8]{inputenc}
\usepackage{amsfonts}
\usepackage{amssymb}
\usepackage{amsmath}
\usepackage[T1]{fontenc}
\usepackage[polish]{babel}
\usepackage{tabto}
\usepackage{mathtools}
\usepackage{hyperref}
\usepackage[parfill]{parskip}  % for after-newline indent

% Wyglad
\DeclareMathSymbol{*}{\mathbin}{symbols}{"01}
\renewcommand*{\thefootnote}{(\arabic{footnote})}
\renewcommand{\bar}{\overline}
\newtheorem{theorem}{Theorem}[section]
\newtheorem{lemma}[theorem]{Lemat}
\TabPositions{1.5em, 3em}

% Logika
\newcommand{\fa}[2]{\forall_{#1 \in #2}\quad}
\newcommand{\faa}[1]{\forall_{\substack{#1}}\quad}
\newcommand{\fe}[2]{\exists_{#1 \in #2}\quad}

% Zbiory
\def\Z{\ensuremath\mathbb{Z}}
\def\N{\ensuremath\mathbb{N}}
\def\R{\ensuremath\mathbb{R}}
\def\Q{\ensuremath\mathbb{Q}}
\def\K{\ensuremath\mathbb{K}}

% GAL
% \newcommand{\norm}[1]{\lVert #1 \rVert}
\newcommand{\scalar}[2]{\langle #1, #2 \rangle}
\def\Kmn{\mathbb{K}^{m, n}}
\def\Prim{( \mathbb{K}^{m, n} )^{\ast}}
\def\tr{\text{tr}}
\def\Lkm{L(\Kmn, \Prim)}

% Aksjomaty AM
\newcommand{\notadd}[1]{#1^{-1, +}}
\newcommand{\notmul}[1]{#1^{-1, *}}

% Sumy
\newcommand{\ssum}[1]{\sum_{n=#1}^{\infty}}

% MD
\newcommand{\desc}[2]{#1^{\underline{#2}}}
\newcommand{\evalat}[3]{#1 \big | _#2^#3}


\begin{document}

\textbf{Zadanie:} \hfill \textbf{Autor:} \\ 
\href{http://smurf.mimuw.edu.pl/node/1013}{Sumy / 1.a)} Oblicz sumę: $\sum_{0 \leq i < n}(-1)^ii^3$ \hfill Paweł Balawender \\ 
\rule{\textwidth}{1pt}
\bigskip

\textbf{Sposób I: grupowanie po dwa wyrazy} \\ Oznaczmy interesującą nas sumę przez $S_n$, tj. niech:
\[S_n = \sum_{0 \leq i < n}(-1)^i i^3\]

W szczególności mamy więc:
\begin{align*}
S_3 &= 0 - 1 + 8 \tag{1a}\label{1a} \\
S_4 &= 0 - 1 + 8 - 27 \tag{1b}\label{1b} \\
S_5 &= 0 - 1 + 8 - 27 + 64 \tag{1c}\label{1c} \\
\end{align*}

I w ogólności dla wszystkich $n \in \N$:
\[S_{2n+1} = S_{2n} + (2n+1)^3\]

Zatem znając zwartą postać $S_n$ dla $n$ parzystych, będziemy w stanie łatwo ją uogólnić do wszystkich $n \in \N$. Załóżmy zatem, że $2 \mid n$, tj. $n = 2k$ dla pewnego $k \in \N$. Wtedy:
\[S_n = S_{2k} = \sum_{0 \leq i < 2k}(-1)^i i^3 = \sum_{i=0}^{2k-1}(-1)^i i^3\]

Patrząc na równości \ref{1a}-\ref{1c} możemy spostrzec (dowód pominę), że dla $n$ parzystych, wyrazy tej sumy możemy pogrupować po dwa. Zatem:
\begin{align*}
S_{2k} &= \sum_{i=0}^{k-1} (2i)^3 - (2i+1)^3 \\
&=\sum_{i=0}^{k-1} 8i^3 - (8i^3 + 3*4i^2 + 3*2i + 1) \\
&=\sum_{i=0}^{k-1} 8i^3 - 8i^3 - 12i^2 - 6i - 1 \\
&=- \sum_{i=0}^{k-1} 12i^2 + 6i + 1 \\
&=-12\sum_{i=0}^{k-1}i^2 -6\sum_{i=0}^{k-1}i -\sum_{i=0}^{k-1}1
\end{align*}

Możemy teraz znaleźć zwarte postaci powyższych trzech składników, posługując się rachunkiem różnicowym:
%\href{https://www.wolframalpha.com/input/?i=sum+1\%2C+i\%3D0+to+\%28n-1\%29
%https://www.wolframalpha.com/input/?i=sum+i%2C+i%3D0+to+%28n-1%29
%https://www.wolframalpha.com/input/?i=sum+i%5E2%2C+i%3D0+to+%28n-1%29
\begin{align*}
\sum_{i=0}^{n-1}1
&= \sum_{i=0}^{n-1} \desc{i}{0}
= \evalat{i}{0}{n}
= n \\
\sum_{i=0}^{n-1}i
&= \sum_{i=0}^{n-1} \desc{i}{1} 
= \frac{\desc{i}{2}}{2}\big|_0^n 
= \frac{n(n-1)}{2} \\
\sum_{i=0}^{n-1}i^2 
&= \sum_{i=0}^{n-1}\desc{i}{2} + \desc{i}{1} 
= \evalat{\frac{\desc{i}{3}}{3}}{0}{n} +  \evalat{\frac{\desc{i}{2}}{2}}{0}{n} 
= \frac{n(n-1)(2n-1)}{6} \\
\end{align*}


A zatem:
%https://www.wolframalpha.com/input/?i=%28sum+%28-1%29%5Ei+i%5E3%2C+i%3D0+to+%282n-1%29%29+%3D+n%5E2%283-4n%29
\[S_{2k} = -12 \frac{k(k-1)(2k-1)}{6} -6\frac{k(k-1)}{2} - k\]
\[=-2(k^2-k)(2k-1) -3(k^2-k) - k\]
\[=-2(2k^3 - k^2 - 2k^2 + k) -3k^2 +3k - k\]
\[=-4k^3 +2k^2 +4k^2 -2k -3k^2 +3k - k\]
\[=-4k^3 +3k^2\]
\[=k^2(3 - 4k)\]

I wtedy:
%https://www.wolframalpha.com/input/?i=%28sum+%28-1%29%5Ei+i%5E3%2C+i%3D0+to+%282n%29%29+%3D+n%5E2%283%2B4n%29
\[S_{2k+1} = -4k^3 +3k^2 + (2k)^3\]
\[=-4k^3+3k^2 +8k^3\]
\[=4k^3+3k^2\]
\[=k^2(3+4k)\]
A zatem po prostu, dla $k = \lfloor{\frac{n}{2}}\rfloor$:
\[S_n = k^2 (3 + 4 (-1)^{n+1}k)\]
Wynik ten możemy przetestować numerycznie:
\begin{verbatim}
$ cat md.py
#!/usr/bin/python3

# \sum_{i=0}^{n-1} (-1)^i * (i^3)
def real(n):
    return sum( (-1)**i * (i ** 3) for i in range(n))

# k^2 * (3 + 4k * (-1)^(n+1)), k = floor(n/2)
def my(n):
    k = n // 2
    return k*k*(3+4*k*(-1)**(n+1))

limit = 1000
for i in range(limit):
    assert real(i) == my(i)

print('Formula is correct for n <', limit)

$ python3 md.py
Formula is correct for n < 1000
\end{verbatim}



\rule{\textwidth}{1pt}
\bigskip

\textbf{Sposób II: sumowanie przez części} \\
Zastąpmy w docelowej sumie 'zwykłą' potęgę potęgami kroczącymi:
\begin{align*}
\sum (-1)^x x^3 &= \sum (-1)^x (\desc{x}{3} + 3\desc{x}{2} + \desc{x}{1}) \\
&= \sum (-1)^x \desc{x}{3} + 3\sum (-1)^x \desc{x}{2} + \sum (-1)^x \desc{x}{1}
\end{align*}

Obliczmy te składniki po kolei posługując się sumowaniem przez części:
\[\sum u \Delta v = uv - \sum Ev \Delta u\]
Niech:
\[u(x) = \desc{x}{k}\]
\[\Delta v(x) = (-1)^x\]
Wtedy:
\[\Delta u(x) = k\desc{x}{k-1}\]
\[v(x) = \frac{(-1)^{x+1}+1}{2}\]
Istotnie:
\[\Delta v(x) = v(x+1) - v(x) = \frac{(-1)^{x+2}+1 - (-1)^{x+1}-1}{2}\]
\[=\frac{(-1)^x + (-1)^x}{2} = (-1)^x\]
Od razu obliczmy obraz $v(x)$ względem operatora przesunięcia:
\[Ev(x) = \frac{(-1)^x+1}{2}\]
A zatem, za stałą sumowania $c$ przyjmując 0:
\begin{align*}
\sum (-1)^x &= \sum \Delta v \\
&= v +c\\
&= \frac{(-1)^{x+1}+1}{2}
\end{align*}

Oraz dla $k \geqslant 1$:
\begin{align*}
\sum (-1)^x \desc{x}{k} &= \sum u \Delta v  \\
&= uv - \sum Ev \Delta u  \\
&= uv - \frac{k}{2} \sum ((-1)^x+1) \desc{x}{k-1} \\
&=uv - \frac{k}{2}\sum (-1)^x\desc{x}{k-1} - \frac{k}{2}\sum(\desc{x}{k-1}) \\
&=uv - \frac{k}{2}\sum (-1)^x\desc{x}{k-1} - \frac{\desc{x}{k}}{2} \\
&=\desc{x}{k}\frac{(-1)^{x+1}+1}{2} - \frac{k}{2}\sum (-1)^x\desc{x}{k-1} - \frac{\desc{x}{k}}{2}\\
&=\desc{x}{k} \frac{(-1)^{x+1}}{2} - \frac{k}{2}\sum (-1)^x\desc{x}{k-1}\\
\end{align*}

Postępując rekurencyjnie, przeprowadzamy obliczenia dla $k=1,2,3$, by otrzymać:
\begin{align*}
\sum (-1)^x &= \frac{(-1)^{x+1}+1}{2}\\
\sum (-1)^x \desc{x}{1} &= x\frac{(-1)^{x+1}}{2}-\frac{1}{2}\frac{(-1)^{x+1}+1}{2}\\
\sum (-1)^x \desc{x}{2} &= \desc{x}{2}\frac{(-1)^{x+1}}{2} - x\frac{(-1)^{x+1}}{2} + \frac{1}{2}\frac{(-1)^{x+1}+1}{2}\\
\sum (-1)^x \desc{x}{3} &=\desc{x}{3}\frac{(-1)^{x+1}}{2}-\frac{3}{2}\desc{x}{2}\frac{(-1)^{x+1}}{2}+\frac{3}{2}x\frac{(-1)^{x+1}}{2}-\frac{3}{4}\frac{(-1)^{x+1} + 1}{2}\\
\end{align*}

Wtedy, przeprowadzając rachunek:
\[\sum (-1)^x x^3 = \sum (-1)^x \desc{x}{3} + 3\sum (-1)^x \desc{x}{2} + \sum (-1)^x \desc{x}{1}\]
Otrzymujemy:

\[\sum (-1)^x x^3 = \desc{x}{3}\frac{(-1)^{x+1}}{2} + \frac{3}{2}\desc{x}{2}\frac{(-1)^{x+1}}{2} - \frac{1}{2}x\frac{(-1)^{x+1}}{2} + \frac{1}{4}\frac{(-1)^{x+1} + 1}{2}\]

Czyli:
\begin{align*}
\sum_{0 \leq i < n} (-1)^i i^3 &= (-1)^{n+1}n^2(\frac{n}{2}-\frac{3}{4}) + \frac{(-1)^{n+1}+1}{8} \\
&=\frac{1}{8}((-1)^{n+1}4n^3-6n^2(-1)^{n+1}+(-1)^{n+1}+1)
\end{align*}

\end{document}

